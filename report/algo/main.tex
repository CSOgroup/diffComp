\documentclass{article}
\usepackage{amssymb}
\usepackage{amsmath}
\usepackage{graphicx}



\title{Compartment repositioning detection algorithm}
\author{}

\begin{document}
\maketitle

\section{Calder genome segmentation}
A chromosome is a ordered set of positions $C = \{p_1,p_2,...,p_N\}$, where $N$ is the total number of positions. Each position is also called \textit{bin} and its size in base pairs (\textit{bin size}) is decided \textit{a-priori}.


Through the Calder algorithm, a chromosome can be partitioned into a set $\mathcal{D}_C$ of \textit{compartment domains}, which are disjoint sets of adjacent bins. If we assume that each bin can be assigned to a compartment domain (which in practice can be false, given lack of data from the Hi-C), then compartment domains are a partition of $C$.

It is easy to prove that the total number of possible partitions of $C$ in compartment domains is $2^{N - 1}$

Calder also ranks the compartment domains of each chromosome by imposing a order on $\mathcal{D}_C$. This means that, given two domains $d_1$ and $d_2$ belonging to $\mathcal{D}_C$, either $d_1 < d_2$ or $d_1 > d_2$.

We can therefore write:
\begin{align*}
	\mathcal{Y}^C = Calder(C) = \left( \mathcal{D}_C, < \right)
\end{align*}

\subsection{Properties of $\mathcal{Y}^C$}

\begin{figure}[tb]
	\centering
	\includegraphics[width=\textwidth]{data/analysis/new_implementation/samples_bounds.pdf}
	\caption{Example of compartment domain partitioning in some Hi-C samples.}
	\label{fig:partitioning}
\end{figure}


The order on $\mathcal{D}_C$ can be used to assign a rank to each compartment domain, such that

\begin{align*}
	r(d) = \frac{\text{Rank of domain } d}{|\mathcal{D}_C|}
\end{align*}

where $r(d)\in [0, 1]$.

Going to the bin level, we can simply assign a rank to each bin $b$ based on the compartment domain $d$ they belong to ($b\in d$)

\begin{align*}
	r(b) = r(d)
\end{align*}




\section{Compartment repositioning}
Let's now consider two experiments $E_1, E_2$. They will produce two Calder segmentations and rankings $\mathcal{Y}^{C}_1, \mathcal{Y}^{C}_2$.





\end{document}
